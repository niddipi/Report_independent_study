%% bare_conf.tex
%% V1.4b
%% 2015/08/26
%% by Michael Shell
%% See:
%% http://www.michaelshell.org/
%% for current contact information.
%%
%% This is a skeleton file demonstrating the use of IEEEtran.cls
%% (requires IEEEtran.cls version 1.8b or later) with an IEEE
%% conference paper.
%%
%% Support sites:
%% http://www.michaelshell.org/tex/ieeetran/
%% http://www.ctan.org/pkg/ieeetran
%% and
%% http://www.ieee.org/

%%*************************************************************************
%% Legal Notice:
%% This code is offered as-is without any warranty either expressed or
%% implied; without even the implied warranty of MERCHANTABILITY or
%% FITNESS FOR A PARTICULAR PURPOSE! 
%% User assumes all risk.
%% In no event shall the IEEE or any contributor to this code be liable for
%% any damages or losses, including, but not limited to, incidental,
%% consequential, or any other damages, resulting from the use or misuse
%% of any information contained here.
%%
%% All comments are the opinions of their respective authors and are not
%% necessarily endorsed by the IEEE.
%%
%% This work is distributed under the LaTeX Project Public License (LPPL)
%% ( http://www.latex-project.org/ ) version 1.3, and may be freely used,
%% distributed and modified. A copy of the LPPL, version 1.3, is included
%% in the base LaTeX documentation of all distributions of LaTeX released
%% 2003/12/01 or later.
%% Retain all contribution notices and credits.
%% ** Modified files should be clearly indicated as such, including  **
%% ** renaming them and changing author support contact information. **
%%*************************************************************************


% *** Authors should verify (and, if needed, correct) their LaTeX system  ***
% *** with the testflow diagnostic prior to trusting their LaTeX platform ***
% *** with production work. The IEEE's font choices and paper sizes can   ***
% *** trigger bugs that do not appear when using other class files.       ***                          ***
% The testflow support page is at:
% http://www.michaelshell.org/tex/testflow/

\newcommand\Tau{\mathcal{T}}
\documentclass[conference]{IEEEtran}
% Some Computer Society conferences also require the compsoc mode option,
% but others use the standard conference format.
%
% If IEEEtran.cls has not been installed into the LaTeX system files,
% manually specify the path to it like:
% \documentclass[conference]{../sty/IEEEtran}





% Some very useful LaTeX packages include:
% (uncomment the ones you want to load)


% *** MISC UTILITY PACKAGES ***
%
%\usepackage{ifpdf}
% Heiko Oberdiek's ifpdf.sty is very useful if you need conditional
% compilation based on whether the output is pdf or dvi.
% usage:
% \ifpdf
%   % pdf code
% \else
%   % dvi code
% \fi
% The latest version of ifpdf.sty can be obtained from:
% http://www.ctan.org/pkg/ifpdf
% Also, note that IEEEtran.cls V1.7 and later provides a builtin
% \ifCLASSINFOpdf conditional that works the same way.
% When switching from latex to pdflatex and vice-versa, the compiler may
% have to be run twice to clear warning/error messages.






% *** CITATION PACKAGES ***
%
\usepackage{cite}
% cite.sty was written by Donald Arseneau
% V1.6 and later of IEEEtran pre-defines the format of the cite.sty package
% \cite{} output to follow that of the IEEE. Loading the cite package will
% result in citation numbers being automatically sorted and properly
% "compressed/ranged". e.g., [1], [9], [2], [7], [5], [6] without using
% cite.sty will become [1], [2], [5]--[7], [9] using cite.sty. cite.sty's
% \cite will automatically add leading space, if needed. Use cite.sty's
% noadjust option (cite.sty V3.8 and later) if you want to turn this off
% such as if a citation ever needs to be enclosed in parenthesis.
% cite.sty is already installed on most LaTeX systems. Be sure and use
% version 5.0 (2009-03-20) and later if using hyperref.sty.
% The latest version can be obtained at:
% http://www.ctan.org/pkg/cite
% The documentation is contained in the cite.sty file itself.
\usepackage{hyperref}
\hypersetup{
    colorlinks=true,
    linkcolor=blue,
    filecolor=blue,      
    citecolor=blue,
    urlcolor=blue,
    pdftitle={Sharelatex Example},
    bookmarks=true,
    pdfpagemode=FullScreen,
}

% *** GRAPHICS RELATED PACKAGES ***
%
\ifCLASSINFOpdf
   \usepackage[pdftex]{graphicx}
   %declare the path(s) where your graphic files are
   \graphicspath{{../pdf/}{../jpeg/}}
  % and their extensions so you won't have to specify these with
  % every instance of \includegraphics
   \DeclareGraphicsExtensions{.pdf,.jpeg,.png}
   %\usepackage{showframe}
   \usepackage[export]{adjustbox}
\else
  % or other class option (dvipsone, dvipdf, if not using dvips). graphicx
  % will default to the driver specified in the system graphics.cfg if no
  % driver is specified.
  % \usepackage[dvips]{graphicx}
  % declare the path(s) where your graphic files are
  % \graphicspath{{../eps/}}
  % and their extensions so you won't have to specify these with
  % every instance of \includegraphics
  % \DeclareGraphicsExtensions{.eps}
\fi
% graphicx was written by David Carlisle and Sebastian Rahtz. It is
% required if you want graphics, photos, etc. graphicx.sty is already
% installed on most LaTeX systems. The latest version and documentation
% can be obtained at: 
% http://www.ctan.org/pkg/graphicx
% Another good source of documentation is "Using Imported Graphics in
% LaTeX2e" by Keith Reckdahl which can be found at:
% http://www.ctan.org/pkg/epslatex
%
% latex, and pdflatex in dvi mode, support graphics in encapsulated
% postscript (.eps) format. pdflatex in pdf mode supports graphics
% in .pdf, .jpeg, .png and .mps (metapost) formats. Users should ensure
% that all non-photo figures use a vector format (.eps, .pdf, .mps) and
% not a bitmapped formats (.jpeg, .png). The IEEE frowns on bitmapped formats
% which can result in "jaggedy"/blurry rendering of lines and letters as
% well as large increases in file sizes.
%
% You can find documentation about the pdfTeX application at:
% http://www.tug.org/applications/pdftex





% *** MATH PACKAGES ***
%
\usepackage{amsmath}
\usepackage{amssymb}
% A popular package from the American Mathematical Society that provides
% many useful and powerful commands for dealing with mathematics.
%
% Note that the amsmath package sets \interdisplaylinepenalty to 10000
% thus preventing page breaks from occurring within multiline equations. Use:
%\interdisplaylinepenalty=2500
% after loading amsmath to restore such page breaks as IEEEtran.cls normally
% does. amsmath.sty is already installed on most LaTeX systems. The latest
% version and documentation can be obtained at:
% http://www.ctan.org/pkg/amsmath





% *** SPECIALIZED LIST PACKAGES ***
%
%\usepackage{algorithmic}
% algorithmic.sty was written by Peter Williams and Rogerio Brito.
% This package provides an algorithmic environment fo describing algorithms.
% You can use the algorithmic environment in-text or within a figure
% environment to provide for a floating algorithm. Do NOT use the algorithm
% floating environment provided by algorithm.sty (by the same authors) or
% algorithm2e.sty (by Christophe Fiorio) as the IEEE does not use dedicated
% algorithm float types and packages that provide these will not provide
% correct IEEE style captions. The latest version and documentation of
% algorithmic.sty can be obtained at:
% http://www.ctan.org/pkg/algorithms
% Also of interest may be the (relatively newer and more customizable)
% algorithmicx.sty package by Szasz Janos:
% http://www.ctan.org/pkg/algorithmicx




% *** ALIGNMENT PACKAGES ***
%
%\usepackage{array}
% Frank Mittelbach's and David Carlisle's array.sty patches and improves
% the standard LaTeX2e array and tabular environments to provide better
% appearance and additional user controls. As the default LaTeX2e table
% generation code is lacking to the point of almost being broken with
% respect to the quality of the end results, all users are strongly
% advised to use an enhanced (at the very least that provided by array.sty)
% set of table tools. array.sty is already installed on most systems.The
% latest version and documentation can be obtained at:
% http://www.ctan.org/pkg/array


% IEEEtran contains the IEEEeqnarray family of commands that can be used to
% generate multiline equations as well as matrices, tables, etc., of high
% quality.




% *** SUBFIGURE PACKAGES ***
\ifCLASSOPTIONcompsoc
  \usepackage[caption=false,font=normalsize,labelfont=sf,textfont=sf]{subfig}
\else
  \usepackage[caption=false,font=footnotesize]{subfig}
\fi
% subfig.sty, written by Steven Douglas Cochran, is the modern replacement
% for subfigure.sty, the latter of which is no longer maintained and is
% incompatible with some LaTeX packages including fixltx2e. However,
% subfig.sty requires and automatically loads Axel Sommerfeldt's caption.sty
% which will override IEEEtran.cls' handling of captions and this will result
% in non-IEEE style figure/table captions. To prevent this problem, be sure
% and invoke subfig.sty's "caption=false" package option (available since
% subfig.sty version 1.3, 2005/06/28) as this is will preserve IEEEtran.cls
% handling of captions.
% Note that the Computer Society format requires a larger sans serif font
% than the serif footnote size font used in traditional IEEE formatting
% and thus the need to invoke different subfig.sty package options depending
% on whether compsoc mode has been enabled.
%
% The latest version and documentation of subfig.sty can be obtained at:
% http://www.ctan.org/pkg/subfig




% *** FLOAT PACKAGES ***
%
%\usepackage{fixltx2e}
% fixltx2e, the successor to the earlier fix2col.sty, was written by
% Frank Mittelbach and David Carlisle. This package corrects a few problems
% in the LaTeX2e kernel, the most notable of which is that in current
% LaTeX2e releases, the ordering of single and double column floats is not
% guaranteed to be preserved. Thus, an unpatched LaTeX2e can allow a
% single column figure to be placed prior to an earlier double column
% figure.
% Be aware that LaTeX2e kernels dated 2015 and later have fixltx2e.sty's
% corrections already built into the system in which case a warning will
% be issued if an attempt is made to load fixltx2e.sty as it is no longer
% needed.
% The latest version and documentation can be found at:
% http://www.ctan.org/pkg/fixltx2e


%\usepackage{stfloats}
% stfloats.sty was written by Sigitas Tolusis. This package gives LaTeX2e
% the ability to do double column floats at the bottom of the page as well
% as the top. (e.g., "\begin{figure*}[!b]" is not normally possible in
% .TeX2e). It also provides a command:
%\fnbelowfloat
% to enable the placement of footnotes below bottom floats (the standard
% LaTeX2e kernel puts them above bottom floats). This is an invasive package
% which rewrites many portions of the LaTeX2e float routines. It may not work
% with other packages that modify the LaTeX2e float routines. The latest
% version and documentation can be obtained at:
% http://www.ctan.org/pkg/stfloats
% Do not use the stfloats baselinefloat ability as the IEEE does not allow
% \baselineskip to stretch. Authors submitting work to the IEEE should note
% that the IEEE rarely uses double column equations and that authors should try
% to avoid such use. Do not be tempted to use the cuted.sty or midfloat.sty
% packages (also by Sigitas Tolusis) as the IEEE does not format its papers in
% such ways.
% Do not attempt to use stfloats with fixltx2e as they are incompatible.
% Instead, use Morten Hogholm'a dblfloatfix which combines the features
% of both fixltx2e and stfloats:
%
% \usepackage{dblfloatfix}
% The latest version can be found at:
% http://www.ctan.org/pkg/dblfloatfix




% *** PDF, URL AND HYPERLINK PACKAGES ***
%
%\usepackage{url}
% url.sty was written by Donald Arseneau. It provides better support for
% handling and breaking URLs. url.sty is already installed on most LaTeX
% systems. The latest version and documentation can be obtained at:
% http://www.ctan.org/pkg/url
% Basically, \url{my_url_here}.




% *** Do not adjust lengths that control margins, column widths, etc. ***
% *** Do not use packages that alter fonts (such as pslatex).         ***
%There should be no need to do such things with IEEEtran.cls V1.6 and later.
% (Unless specifically asked to do so by the journal or conference you plan
% to submit to, of course. )


% correct bad hyphenation here
\hyphenation{op-tical net-works semi-conduc-tor}


\begin{document}
%
% paper title
% Titles are generally capitalized except for words such as a, an, and, as,
% at, but, by, for, in, nor, of, on, or, the, to and up, which are usually
% not capitalized unless they are the first or last word of the title.
% Linebreaks \\ can be used within to get better formatting as desired.
% Do not put math or special symbols in the title.
\title{Implementation of Decentralized Coordination\\ for Spatial Task Allocation
and Scheduling\\ in Heterogeneous Teams using simulation robots}


% author names and affiliations
% use a multiple column layout for up to three different
% affiliations
\author{\IEEEauthorblockN{Neelesh Iddipilla}
\IEEEauthorblockA{Computer Science Department\\
Oklahoma State University\\
Stillwater\\}
\and 
\IEEEauthorblockN{Dr. Christopher Crick}
\IEEEauthorblockA{Computer Science Department\\
Oklahoma State University\\
Stillwater\\}}

% conference papers do not typically use \thanks and this command
% is locked out in conference mode. If really needed, such as for
% the acknowledgment of grants, issue a \IEEEoverridecommandlockouts
% after \documentclass

% for over three affiliations, or if they all won't fit within the width
% of the page, use this alternative format:
% 
% make the title area
\maketitle

% As a general rule, do not put math, special symbols or citations
% in the abstract
\begin{abstract}
It is crucial to show that a system is adaptable on actual robots along with performing emperical sensitivity analysis and showing that a proposed system works. In our paper we have tested the theoritically proven system on a heterogeneous team of robots in a simulation environment and presented the results. A mixed integer linear formulation\cite{feo2016decentralized} is used to solve the problem of decentralization aiming to find the adaptability on robots.Our main contributions include comming up with a solution on how to test the algorithm using a heterogeneous team of simulated robots and showing the results for both the centralized and the decentralized approaches proposed in paper\cite{feo2016decentralized}.  
\end{abstract}

% no keywords

% For peer review papers, you can put extra information on the cover
% page as needed:
% \ifCLASSOPTIONpeerreview
% \begin{center} \bfseries EDICS Category: 3-BBND \end{center}
% \fi
%
% For peerreview papers, this IEEEtran command inserts a page break and
% creates the second title. It will be ignored for other modes.
\IEEEpeerreviewmaketitle



\section{Introduction}
% no \IEEEPARstart
Multiple heterogenous team of robots with coordination and planning are answer to many scenarios where human life could be in danger such as serach and rescue scenarios,survelliance etc. Coordination and scheduling is required to avoid conflict, and optimize performance when executing a collaborative mission.In this work we focus on a system which has been proven robust but not tested on any actual robots.For testing the system on real heterogenous team of robots we need peer to peer communication among all robots to share their missions and completion of tasks.But the available on board communication hardware doesnot support peer to peer communication to test on multi heterogenous team of robots. We have tackled this problem using ros platform by creating a ros node for each robot and for peer to peer communication we used rosmsgs .For simulating the heterogenous team of robots we used gazebo which is an open-source well designed simulator for testing algorithms on various types of robots.The problem we considered to test is a Mixed Integer linear formulation which solves the coordination and planning of heterogenous team of robots in a collaborative mission.
The algorithm consists of a centralized approach and a topdown decentralized scheme derived from the centralized approach.Our main contribution include comming up with a senario on how to test the algorithm using heterogenous team of simulated robots and showing the results obtained for both the centralized and the decentralized approaches.  
% An example of a floating figure using the graphicx package.
% Note that \label must occur AFTER (or within) \caption.
% For figures, \caption should occur after the \includegraphics.
% Note that IEEEtran v1.7 and later has special internal code that
% is designed to preserve the operation of \label within \caption
% even when the captionsoff option is in effect. However, because
% of issues like this, it may be the safest practice to put all your
% \label just after \caption rather than within \caption{}.
%
% Reminder: the "draftcls" or "draftclsnofoot", not "draft", class
% option should be used if it is desired that the figures are to be
% displayed while in draft mode.
%
%\begin{figure}[!t]
%\centering
%\includegraphics[width=2.5in]{myfigure}
% where an .eps filename suffix will be assumed under latex, 
% and a .pdf suffix will be assumed for pdflatex; or what has been declared
% via \DeclareGraphicsExtensions.
%\caption{Simulation results for the network.}
%\label{fig_sim}
%\end{figure}

% Note that the IEEE typically puts floats only at the top, even when this
% results in a large percentage of a column being occupied by floats.


% An example of a double column floating figure using two subfigures.
% (The subfig.sty package must be loaded for this to work.)
% The subfigure \label commands are set within each subfloat command,
% and the \label for the overall figure must come after \caption.
% \hfil is used as a separator to get equal spacing.
% Watch out that the combined width of all the subfigures on a 
% line do not exceed the text width or a line break will occur.
%
\iffalse
\begin{figure*}[!t]
\centering
\subfloat[1]{\includegraphics[width=0.5,textwidth]{/home/neelesh/Independent_Study/images/homogeneous.jpg}}
\hfil
\subfloat[2]{\includegraphics[width=0.5,textwidth]{/home/neelesh/Independent_Study/images/heterogeneous.jpg}
\label{fig_second_case}}
\caption{Task Efficiency of each robot (a) homogeneous case and (b) heterogeneous case}
\label{fig_sim}
\subfloat[1]{\includegraphics[width=0.5,textwidth]{/home/neelesh/Independent_Study/images/homogeneous1.jpg}
\label{fig_first_case1}}
\hfil
\subfloat[2]{\includegraphics[width=0.5,textwidth]{/home/neelesh/Independent_Study/images/heterogeneous1.jpg}
\label{fig_second_case1}}
\caption{Task Efficiency of each robot (a) homogeneous case and (b) heterogeneous case}
\label{fig_sim1}
\end{figure*}
\fi
%
% Note that often IEEE papers with subfigures do not employ subfigure
% captions (using the optional argument to \subfloat[]), but instead will
% reference/describe all of them (a), (b), etc., within the main caption.
% Be aware that for subfig.sty to generate the (a), (b), etc., subfigure
% labels, the optional argument to \subfloat must be present. If a
% subcaption is not desired, just leave its contents blank,
% e.g., \subfloat[].


% An example of a floating table. Note that, for IEEE style tables, the
% \caption command should come BEFORE the table and, given that table
% captions serve much like titles, are usually capitalized except for words
% such as a, an, and, as, at, but, by, for, in, nor, of, on, or, the, to
% and up, which are usually not capitalized unless they are the first or
% last word of the caption. Table text will default to \footnotesize as
% the IEEE normally uses this smaller font for tables.
% The \label must come after \caption as always.
%
%\begin{table}[!t]
%% increase table row spacing, adjust to taste
%\renewcommand{\arraystretch}{1.3}
% if using array.sty, it might be a good idea to tweak the value of
% \extrarowheight as needed to properly center the text within the cells
%\caption{An Example of a Table}
%\label{table_example}
%\centering
%% Some packages, such as MDW tools, offer better commands for making tables
%% than the plain LaTeX2e tabular which is used here.
%\begin{tabular}{|c||c|}
%\hline
%One & Two\\
%\hline
%Three & Four\\
%\hline
%\end{tabular}
%\end{table}


% Note that the IEEE does not put floats in the very first column
% - or typically anywhere on the first page for that matter. Also,
% in-text middle ("here") positioning is typically not used, but it
% is allowed and encouraged for Computer Society conferences (but
% not Computer Society journals). Most IEEE journals/conferences use
% top floats exclusively. 
% Note that, LaTeX2e, unlike IEEE journals/conferences, places
% footnotes above bottom floats. This can be corrected via the
% \fnbelowfloat command of the stfloats package.

% conference papers do not normally have an appendix


% use section* for acknowledgment
\section{Related Work}
A variety of work investigates the coordination and scheduling of multi-robot missions other than Mixed Integer problem formulation approach in terms of both centralization and decentralization architectures.Some of them proposed the probabilistic model for a multi-robot grid i;e using Markov Decision Process for spatial task allocation problems\cite{claes2015effective}\cite{di2011decentralized}\cite{ponda2015cooperative}.Expectation maximization is used in a project for detection of plan deviation for multi-agent cordination in a collaborative mission\cite{claes2015effective}\cite{banerjee2016detection}.Heuristic based models are also used for the clustering and opportunistic path planning, to perform a bounded search of possible time-extended schedules and allocations\cite{jones2011time}\cite{maza2011distributed}\cite{jones2011time}.
 Few projects has tackled the problem of coordination of multi agents in simulation using Contract-Netprotocol\cite{lemaire2004distributed} and multiagent-based prototype system that uses swarming techniques\cite{dasgupta2008multiagent}.Some papers provided formal proofs for alogorithms solving multiple agent task assignment to acheive colloborative mission, such as inapproximability results for connectivity constrained multi robot planning with periodic connectivity\cite{hollinger2012multirobot} and consensus-based bundle algorithm (CBBA)\cite{choi2009consensus}\cite{ponda2010decentralized}.Centralized and Decentralized approaches have been proposed and simulation results for their approaches have been shown proving the approaches work, But they have not tested on any simulated or actual robots\cite{feo2016decentralized}\cite{flushing2014mathematical}\cite{galceran2013survey}.
Several multi-robot projects are led in different research teams , but up to now few multi-UAVs or multi-Robots(CAR) applications have already been demonstrated\cite{gancet2005task}\cite{garzon2016multirobot}. Most of them mainly rely only for unique set of robots such as UAVs or ground robots such as robotic cars. And also most of them use a combination of both wireless networks and cameras to maintian coordination. In our work we focus on use of heterogenous team of robots which consists of both ground (turtlebots) and aerial (quadcopters) robots. The robots which we used are simulated in Gazebo simulator. A few projects have used heterogeneous team of robots for testing their  distributed task allocation frameworks \cite{ponda2010decentralized} \cite{sariel2011generic}\cite{shiroma2009comutar}.But the architectures are different from what we are using(MIP). 
\section{SPATIAL TASK ALLOCATION and SCHEDULING IN HETEROGENEOUS TEAMS}
In this section, the we explained the concepts that underlie the model of the mission.Then provide a concise statement of the STASP-HMR. Along with that we have presented the models of which we have used for testing the mathematical approach on simulated robots. The concepts and testing models we used are explained for both the centralized and decentralized architectures. 
\subsection{Mission Representation}
Let A be the team of heterogeneous agents available to perform joint mission in an environment of specified dimension. Overall mission is decomposed into a set of tasks \( \Tau \) The tasks can be non atomic,incrementally providing a reward proportional to progress acheived in their completion and can eventually be brought to an en.The spatial layout of tasks is captured by a traversability graph that defines how agents move between tasks\cite{feo2016decentralized}.
The graph can be represented as G = ( \( \Tau \) , E ) ,where E contains an arc (i,j) if task j can be scheduled right after task i. In general case graph G is complete (i.e.\, E = \( \Tau \) \( \times \) \( \Tau \)), and in this we assume all tasks are independent of each other.

From the point of view of the mission, the complete execution of any task \( \tau \) \( \epsilon \) \( \Tau \) provides an overall utility, or reward, indicated with \( R_{\tau} \).

\subsection{Task Efficiency Model}
Task efficiency model as the name suggests is the efficiency with which an agent can perform a specified task. The intution behind this is that any progress on the completion of a task is proportional to overall time devoted to it. In a given time if an agent performs much faster then it is considered more efficient \( \psi \) : A \( \times \) \( \Tau \) \( \to \) {R}\cite{feo2016decentralized}. 

\section{CENTRALIZED SOLUTION APPROACH}
We formulate the STASP-HMR as stated above by means of a mixed-integer linear program (MIP). An optimal solution to the MIP defines plans for each one of the agents with the goal of maximizing the total mission reward.

\begin{align}
&\displaystyle \max_{\tau \in \Tau}\hspace{0.5cm}R_\tau \phi_\tau \\ \notag\\
&\displaystyle \text{subject to} \notag \\
&\displaystyle \sum_{(0,j) \in E} {x_{0j}}^k  =  1\hspace{2.6cm}  {k \in A}\\ 
&\displaystyle \sum_{(i,0) \in E} {x_{i0}}^k  =  1 \hspace{2.6cm} {k \in A}\\ 
&\displaystyle \sum_{(i,j) \in E} {x_{ij}}^k  = \sum_{(j,i) \in E} {x_{ji}}^k = {y_j}^k \hspace{0.2cm} {k \in A,j \in \Tau}\\ 
&\displaystyle{{t_i}^k}+{{w_i}^k}-{{t_j}^k}\leq{(1-{{x_{ij}}^k})T} \hspace{0.4cm}{k \in A,(i,j) \in E,i,j \ne 0}\\
&{{y_i}^k} \leq {{t_i}^k},{{w_i}^k} \leq T{{y_i}^k}\hspace{1.6cm}  k \in A,i \in \Tau\\
&\phi_{\tau} \leq \sum_{k \in A} {\psi_k}(\tau){{w_i}^k} \hspace{2.0cm} k \in A,\tau \in \Tau\\
& 0 \leq \phi_{\tau} \leq C_{m}(\tau) \hspace{2.3cm} \tau \in \Tau \\
&\phi_{\tau} \in \mathbb{R} \\
&{{t_i}^k},{{w_i}^k} \in \mathbb{N}\hspace{3cm} k \in A,i \in \Tau\\
&{x_{ij}}^k,{y_i}^k \in \{0,1\} \hspace{2.3cm}k \in A,i,j \in \Tau
\end{align}
We use the following decision variables to build the MIP model for the STASP-HMR presented in (1)-(11):
\begin{align*}
&{x_{ij}}^k:\text{binary, equals 1 if agent k traverses arc (i,j)} \in E;\\
&{y_{i}}^k:\text{binary, equals 1 if agent k is assigned to task i } \in \Tau;\\
&\phi_{\tau}:\text{service provided to task }\tau \in \Tau \text{ by all agents};\\
&{{t_i}^k}:\text{starting time assigned to task i} \in \Tau \text{ by agent k};\\
&{{w_i}^k}:\text{time assigned to task i} \in \Tau \text{ for agent k}.\\
\end{align*}

The objective function (1) defines the quality of a mission plan in terms of its utility, quantifying the expected effect of the agents' activities over the current state of completion map \(C_{m}\). A dummy vertex denoted by 0 here represents the starting point and ending point of the agents path. Graph G is extended with arcs from 0 to each one of the tasks that are iniatially accesible. Constraints (2-4) ensure path traversability.Contraints 5 eliminate sub-tours together with (6) they define the bounds of the variables t and w. The completion level of each tasks are bounded by constraints (7-8). These bounds ensure that task \(\tau\)text{ provides a maximum reward equal to \({R_{\tau}C_{m}(\tau)}\), and that the utility of the plan is contributed with \(R_{\tau}\) scaled by the completion of \(\tau\) (i;e \(\phi_{\tau}\)).Finally, constraints (9-11) set the real,integer, and binary requirements on model variables\cite{feo2016decentralized}.
\begin{figure}
\center{\includegraphics[width=0.45\textwidth] {img.png}} 
\caption{\texttt{Model designed for testing centralized approach on heterogeneous team of simulated robots}}\label{cen_model}
\end{figure}

The model \ref{cen_model} is used for testing centralized approach on heterogeneous team of simulated robots.The MIP solves computes optimal mission plans using full knowledge of all agents. The selcted optimal sequence of tasks for each agent are converted in waypoints of the locations in the gazebo world. The computed way points from each ros node are used by the corresponding agents to complete the subtasks.  

\section{DECENTRALIZED SOLUTION APPROACH}
In decentralized approach each agent runs a replica of the mathematical model, based on local data and information sharing.In a decentralized operation, communication among the agents enables the acquisition of relevant information regarding the past, current, and planned activities. This is maintained by a global completion map for all the tasks\cite{feo2016decentralized}.


\begin{figure}
\center{\includegraphics[width=0.45\textwidth] {img1.png}} 
\caption{\texttt{Model designed for testing decentralized approach on heterogeneous team of simulated robots}}\label{decen_model}
\end{figure}

In \ref{decen_model} the model designed for testing decentralized approach each agent computes the optimal set of tasks by running the mathematical approach using local knowledge and information obtained by sharing. Ros messages between rosnodes of each agents are used for information sharing.
\section{Experimental Setup}

The aim of the experiment is to survey area using heterogeneous team of robots. We use heterogeneous robots which includes turtlebots and uavs. We perform the experiment on both centralized and decentralized approaches to compare results.

The experiment is as follows, the total area is divided into multiple cells where each cell is considered as a subtask. Each subtask based on the importance of location or cell has been associated with a reward. If the location is more important to be surveyed by a particular agent then we assigned more reward for that particular agent,for example a subtask with an uneven surface cannot be covered by a turtlebot hence we assign less reward for turtlebots and more reward to uavs for same subtask\cite{flushing2014mathematical}.

Each subtask is also been associated with efficiency .Based on which the model finds the optimal set of tasks for each agent maximizing the total reward and service time combined for all the tasks.

The total area to be surveyed by the agent is divided into 30 cells which are represented via 0 to 29 nodes in the directed traversability graphs.Each \(\tau\) \( \epsilon \) [0 ,1 ,2 ,3 ,4 ,5 ,6 ,7 ,8 ,9 ,10 ,11 ,12 ,13 ,14 ,15 ,16 ,17 ,18 ,19 ,20 ,21 ,22 ,23 ,24 ,25 ,26 ,27 ,28 ,29].
The dummy vertex as mentioned in the formulation helps to recognize the starting and ending points of the selected traversability path. The dummy vertex used is node number 30.
The directed traversability graphs for each agent is same for both centralized and decentralized approach.In Centralized approach the algorithm we solve the algorithm for all agents jointly and assign the best possible sub tasks for each agent.In the decentralized approach we run the replica of algorithm for each agent while considering the information received from other agents.


In Figure (\ref{Cen_agents}) and (\ref{Decen_agents}) the directed traversability graph of each agent is given. Each agent's directed traversability graph is the set of all possible tasks it can perform in a sequence. For example the traversability graph of agent1 has the following edges [(0, 1),(1, 7),(6,12),(12,13),(7, 13),(13, 19),(19, 20),(20, 21),(21,15),(15,14),(14,8),(8,2),(2,1)]. In the above represented edges for task 13 can be performed only after completion of task 7 or 12. Hence task 13 is allocated to agent1 either after task 7 or when task 12 is completed by any of the agents.

For evaluating the experiment on robots we are using Gazebo simulator and ros.We have created thread for each agent to make  them perform tasks simultaneously in the gazebo environment.The linear program tool kit we used for solving the MIP is PULP which is an open source python library. 

we have used networkx python library for creating the directed traversability graphs for each agent.
\section{Results}

 Figure (\ref{simulator}) shows the heterogeneous team of robots consiting of two turtlebots and two iris quadcopter completing assigned tasks simulatenously in the world of gazebo simulator.
Figure (\ref{Cen_agents}) show the directed traversability graphs and the selected one of the best possible sequence of tasks when t = 5 mission intervals. In the selected traversability path of each agent the node 30 is the vertex which is used to identify the starting and ending tasks in selected path. In figure (\ref{Cen_agents}) the selected sequence of tasks is 30\(\rightarrow\)6\(\rightarrow\)12\(\rightarrow\)13\(\rightarrow\)19\(\rightarrow\)20\(\rightarrow\)30. For \(\tau\) = 5 mission intervals, the optimal solution assigned 5 tasksto each agent allocating one mission interval for each task.

Similarly figure (\ref{Decen_agents}) show the traversability graphs and selected paths for all agents using decentralized approach and when look ahead planning t = 5 mission intervals. Even though the directed traversability graphs assigned to each agent is same in both centralized and decentralized algorithms, the assigned task sequences by the optimal solutions are different for some agents.

For example the optimal sequence of tasks assigned by the centralized approach to agent4 is 30 \(\rightarrow\)17\(\rightarrow\)23\(\rightarrow\)29\(\rightarrow\)28\(\rightarrow\)22\(\rightarrow\)30 which can be seen in figure (\ref{simulator}). While the optimal sequence of tasks assigned by the decentralized approach to agent4 is 30\(\rightarrow\)28\(\rightarrow\)22\(\rightarrow\)21\(\rightarrow\)27\(\rightarrow\)26\(\rightarrow\)30 which can be seen in figure (\ref{Decen_agents}).

 Figure (\ref{Cen_results}) show the results obtained for Centralized approach i;e total percaentage of area that when different mission intervals are allocated for each agent and time taken to complete the all the tasks assigned to agents in minutes.

Similarly figure (\ref{De_results}) show the results obtained for Decentralized approach. The total percaentage of area that when different look a head planning mission intervals are assigned for agents. The subfigure also shows time taken for all the agents to complete the tasks at various look a head planning mission intervals. 

Even though the optimal sequence of tasks is different for few agents in both  centralized and decentralized approaches. The total percentage of area covered in both cetralized and decentralized approaches when given same mission intervals is almost similiar. 

The results show that to complete the same number of tasks Decentralized approach performs slightly slower compared to centralized approach with same number of mission intervals provided information exchange between agents.


\section{Conclusion}
The specific objective of this study is to find the adaptability of the proven mathematical approach on actual robots. Our experimental results we got by testing the mathematical approach on heterogeneous team of simulated robots confirms that the both centralized and decentralized approaches works and multi-robots can complete a colloborative mission. 
This research will serve as a base for future studies to investigate the results of centralized and decentralized approaches when tested on real robots.
 
For further reference : \url{https://github.com/niddipi/MILP_to_Collaborative_Missions_with_Heterogeneous_Teams_using_Pulp}.

\begin{figure}
\hfill\begin{minipage}{0.5\textwidth}\centering
\includegraphics[width=1\textwidth,height=0.25\textheight]{/home/neelesh/Independent_study/images/figure_1.png}
\includegraphics[width=1\textwidth,height=0.25\textheight]{/home/neelesh/Independent_study/images/figure_2.png}
\includegraphics[width=1\textwidth,height=0.25\textheight]{/home/neelesh/Independent_study/images/figure_3.png}
\includegraphics[width=1\textwidth,height=0.25\textheight]{/home/neelesh/Independent_study/images/figure_4.png}
\caption{\texttt{Results obtained for Centralized approach}}\label{Cen_agents}
\end{minipage}
\end{figure}

\begin{figure}
\hfill\begin{minipage}{0.5\textwidth}\centering
\includegraphics[width=1\textwidth,height=0.25\textheight]{/home/neelesh/Independent_study/images/de_figure_1.png}
\includegraphics[width=1\textwidth,height=0.25\textheight]{/home/neelesh/Independent_study/images/de_figure_2.png}
\includegraphics[width=1\textwidth,height=0.25\textheight]{/home/neelesh/Independent_study/images/de_figure_3.png}
\includegraphics[width=1\textwidth,height=0.25\textheight]{/home/neelesh/Independent_study/images/de_figure_4.png}
\caption{\texttt{Results obtained for Decentralized approach}}\label{Decen_agents}
\end{minipage}
\end{figure}
\begin{figure}
\hfill\begin{minipage}{0.5\textwidth}\centering
\includegraphics[width=0.8\textwidth,height=0.20\textheight]{/home/neelesh/Independent_study/images/working.jpg}
\caption{\texttt{Heterogeneous robots performing tasks simultaneously}}\label{simulator}
\includegraphics[width=1\textwidth,height=0.20\textheight]{/home/neelesh/Independent_study/images/result1.png}
\includegraphics[width=1\textwidth,height=0.20\textheight]{/home/neelesh/Independent_study/images/result2.png}
\caption{\texttt{Results obtained for Centralized approach}}\label{Cen_results}
\includegraphics[width=1\textwidth,height=0.20\textheight]{/home/neelesh/Independent_study/images/result3.png}
\includegraphics[width=1\textwidth,height=0.20\textheight]{/home/neelesh/Independent_study/images/result4.png}
\caption{\texttt{Results obtained for Decentralized approach}}\label{De_results}
%\caption{\texttt{minipage}}
\end{minipage}
\end{figure}
\newpage
% trigger a \newpage just before the given reference
% number - used to balance the columns on the last page
% adjust value as needed - may need to be readjusted if
% the document is modified later
%\IEEEtriggeratref{8}
%The "triggered" command can be changed if desired:
%\IEEEtriggercmd{\enlargethispage{-5in}}
% references section
% can use a bibliography generated by BibTeX as a .bbl file
% BibTeX documentation can be easily obtained at:
% http://mirror.ctan.org/biblio/bibtex/contrib/doc/
% The IEEEtran BibTeX style support page is at:
% http://www.michaelshell.org/tex/ieeetran/bibtex/
\bibliographystyle{IEEEtran}
% argument is your B.TeX string definitions and bibliography database(s)
\bibliography{latexbib.bib}
%
% <OR> manually copy in the resultant .bbl file
% set second argument of \begin to the number of references
% (used to reserve space for the reference number labels box)

% that's all folks
\end{document}
